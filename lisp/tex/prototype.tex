\documentclass[a4paper,DIV=17,11pt,headings=standardclasses]{scrartcl}
%\usepackage{geometry}
\usepackage[utf8]{inputenc}
\usepackage[T1]{fontenc}
\usepackage{graphicx}
\usepackage[english]{babel}
\usepackage{csquotes}
\usepackage{longtable}



\usepackage{newunicodechar}
\newunicodechar{♭}{$\flat$}
\newunicodechar{♯}{$\sharp$}
\newunicodechar{♮}{$\natural$}

\def\fourLine#1{%
\begin{longtable}{p{0.25\linewidth}p{0.25\linewidth}p{0.25\linewidth}p{0.25\linewidth}}
  \includegraphics[scale=0.5]{../tikz/#1-a.pdf} &
  \includegraphics[scale=0.5]{../tikz/#1-b.pdf} &
  \includegraphics[scale=0.5]{../tikz/#1-c.pdf} &
  \includegraphics[scale=0.5]{../tikz/#1-d.pdf} \\
\end{longtable}
}

\def\twoLine#1{%
\begin{longtable}{p{0.25\linewidth}p{0.25\linewidth}p{0.25\linewidth}p{0.25\linewidth}}
  \includegraphics[scale=0.5]{../tikz/#1-a.pdf} &
  \includegraphics[scale=0.5]{../tikz/#1-b.pdf} &
\end{longtable}
}



\begin{document}
\hspace{1cm}
\vspace{2cm}

\thispagestyle{empty}
\begin{center}
  {\huge Vieltönigkeit -- The Catalogue}\\[1ex]
  \textit{\today}
\end{center}

\vfill

\hrule

\tableofcontents

\newpage

\section{Terminology}
\begin{description}
\item[incitato] A \textit{natura} of an interval. Going from a softer note or hexachord syllable to a harder note.
\item[molle] A \textit{natura} of an interval. Going from a harder note or hexachord syllable to a softer one.
\item[natura] A property of an interval, usually either \textit{incitato} or \textit{molle}. If an intervall is \textit{incitato} in ascending direction, it will be \textit{molle} descending, and vice versa.
\item[ordine naturale] Notes found in the usual Gamut: A, ♮B, ♭B, C, D, E, F and G. Also called \enquote{mano naturale}, as in \enquote{the Guidonian hand}. These are the pitches that can be named using the three usual hexachords: the natural hexachord on C, the soft one on F and the hard one on G.
\item[quarta] Fourth. \textit{Incitato} ascending, \textit{molle} descending. Example: D-G, F-♭B, ♯C-♯F.
\item[quinta] Fifth. \textit{Incitato} ascending, \textit{molle} descending. Example: D-A, ♯G-♯D, ♭A-♭E.
\item[semitono maggiore] Minor second, diatonic semitone, leading note, \enquote{mi-fa}. \textit{Molle} ascending, \textit{incitato} descending. Example: E-F, ♯G-A, D-♭E, ♯E-♯F.
\item[semitono minore] Augmented prime, chromatic semitone. \textit{Incitato} ascending, \textit{molle} descending. Example: C-♯C, ♭A-A.
\item[terza minore] Minor third. \textit{Molle} ascending, \textit{incitato} descending. Example: D-F, ♯F-A, F-♭A.
\item[terza maggiore] Major third. \textit{Incitato} ascending, \textit{molle} descending. Example: C-E, A-♯C, ♭D-F.
\item[tono] Major second, whole step. \textit{Incitato} ascending, \textit{molle} descending. Example: D-E, ♯F-♯G, ♭D-♭E.
\end{description}


\pagebreak

\section{Two-part voiceleadings}
\subsection{\textit{Tono} and \textit{quinta}}
\fourLine{c001}

\subsection{\textit{Tono} and \textit{semitono maggiore}}
\fourLine{c002}

\subsection{\textit{Tono} and \textit{tono}}
\twoLine{c003}
\twoLine{c004}
\twoLine{c005}




\end{document}
